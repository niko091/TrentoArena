\chapter{Il progetto \ProjectTitle}
% Grid with 1 row and 3 columns for images inside the folder slides
\begin{figure}[H]
    \begin{subfigure}{0.33\textwidth}
        \includegraphics[width=\textwidth]{slides/Diapositiva2.PNG}
        \caption{I problemi}
    \end{subfigure}
    \begin{subfigure}{0.33\textwidth}
        \includegraphics[width=\textwidth]{slides/Diapositiva3.PNG}
        \caption{La soluzione}
    \end{subfigure}
    \begin{subfigure}{0.33\textwidth}
        \includegraphics[width=\textwidth]{slides/Diapositiva4.PNG}
        \caption{I vantaggi}
    \end{subfigure}
\end{figure}

\section{I problemi}
    Il progetto mira a risolvere tre problemi:\newline
    Il primo problema riguarda uno scarso senso di comunità sportiva all'interno Comune di Trento, che essendo una città universitaria ospita molte persone con background differenti, molte delle quali praticano sport a livello amatoriale, ma si limitano e incotrarsi in piccoli gruppi o non riescono a trovare altri con la stessa passione.
    \newline
    Il secondo problema riguarda la difficoltà  nell'organizzare incontri sportivi amatoriali, gli strumenti per organizzarsi sono molti, come gruppi su app di messaggistica, pagine sui social, volantini e passaparola. Le strutture sportive pubbliche sono molte, ben attrezzate e distribuite in tutto il territorio. Questo comporta una frammentazione della comunità sportiva.
    \newline
    Il terzo problema riguarda la mancanza di motivazione e costanza per continuare ad allenarsi, quando si inizia a praticare uno sport a livello amatoriale si è motivati dal fatto che si sta iniziando una nuovo percorso, ma col passare del tempo questa tende a svanire, soprattutto senza stimoli esterni e qualcuno coi cui condividere la stessa passione.
\section{La soluzione}
   Il progetto mira a realizzare un'applicazione web che permetta ai cittadini di organizzare incontri sportivi amatoriali per gli sport più popolari nei quali ci sono due giocatori/squadre, nei quali c'è una chiara definizione di vittoria/sconfitta e per cui il comune offre delle strutture pubbliche. Dopo aver effettuato l'accesso, sarà possibile lanciare/partecipare a sfide, visualizzare gli impianti tramite la mappa, inviare richieste di amicizia per vedere le attività degli altri utenti e competere per scalare la classifica globale. Tutte le funzionalità dell'applicazione saranno completamente fruibili tramite browser, senza necessità di installare l'app in locale. L'applicazione offrirà funzionalità come notifiche per eventi creati dai tuoi amici o di tuo interesse.
   
\section{I vantaggi}
    \begin{itemize}
        \item \textbf{Semplcità nel trovare/creare partite:} la possibilità di trovare o creare sfide in modo semplice incoraggia gli utenti a mettersi in gioco e partecipare a partite già esistenti.
        \item \textbf{Gamification:} tramite il meccanismo di punti e livelli (che funziona in modo simile all'ELO negli scacchi) gli utenti sono invogliati a partacipare e a sfidarsi tra di loro per scalare la classifica e aumentare il proprio status virtuale.
        \item \textbf{Visualizzazione deli impianti pubblici:} grazie alla mappa interattiva è possibile visualizzare la posizione e lo stato delle strutture sportive pubbliche, insieme alla lista delle attrezzature presenti e alle sfide che si terranno in quella struttura.
        \item \textbf{Interfaccia intuitiva:} Presentare le informazioni con un'interfaccia grafica, accessibile anche a utenti sprovvisti di conoscenze informatiche, assicura che l'applicazione sarà utile a una vasta quantità di cittadini.
        \item \textbf{raccolta di dati per il comune:} i dati relativi alla frequenza di utilizzo delle strutture pubbliche, all'affluenza nei diversi giorni/orari e alla pololarità dei diversi sport, saranno resi disponibili al comune (in forma anonimizzata).
    \end{itemize}


\section{I limiti dell'applicazione}
    \begin{itemize}
        \item \textbf{Dipendeza da una connesione internet:} per visualizzare le sfide create da altri utenti, lo stato delle strutture e la classifica è necessario essere connessi a internet.
        \item \textbf{Il sistema si basa sull’onestà degli utenti:} la vittoria deve essere registrata nell'app dagli utenti, che quindi dovranno stabilire un vincitore. Questa decisione non può essere presa direttamente dall'applicazione, ma si basa sull'onestà degli utenti.
        \item \textbf{L’incontro tra utenti non conosciuti può esporre a rischi o disagi:} incontrare dal vivo persone conosciute tramite app può portare a disagi/rischi, l'applicazione cerca di raccogliere feedback sul comportamento dei giocatori, coloro che verranno segnalati come non adatti verranno sospesi temporaneamente e dopo un certo numero di sospensioni verranno banditi.
        \item \textbf{L’app richiede un’utenza attiva per mantenere la sua utilità:} l'app richiede una discreta partecipazione da parte degli utenti, in quanto gli eventi vengono creati dalla community.
    \end{itemize}