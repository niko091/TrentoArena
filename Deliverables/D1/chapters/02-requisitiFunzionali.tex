\chapter{Requisiti Funzionali} 
    \section{Requisiti funzionali comuni a tutti gli utenti}
        \begin{rfList}
            \rfItem \textbf{Login:} 
            L'utente deve potersi autenticare per accedere alla piattaforma. Il sistema deve fornire due metodi di accesso:
            \newline
            \textbf {Login Standard:} 
            L'utente inserisce la propria e-mail (o nome utente) e password. Il sistema verifica le credenziali; in caso di successo, reindirizza l'utente alla propria Dashboard
            \newline
            \textbf{Login tramite Google:} 
            L'utente seleziona "Accedi con Google". Il sistema avvia il flusso di autenticazione OAuth2. Se l'utente Google è già associato a un account della piattaforma, il sistema lo autentica e lo reindirizza alla propria dashboard, altrimenti vedi RF2.

            \rfItem \textbf{Registrazione:} 
            Il sistema deve permettere a un nuovo utente di creare un account. Sono previsti due flussi:
            \newline
            \textbf {Registrazione Standard:} 
            L'utente compila un modulo che richiede: nome utente (che deve essere unico nel sistema), e-mail, password e conferma password.
            Il sistema valida i dati (unicità del nome utente, validità dell'email, corrispondenza delle password). Il sistema crea l'account in stato "non confermato" e invia un'e-mail di conferma all'indirizzo fornito. L'utente deve cliccare sul link nell'e-mail per attivare l'account
            \newline
            \textbf{Registrazione tramite Google:} 
            L'utente seleziona "Registrati con Google".
            Il sistema recupera l'e-mail e il nome dall'account Google.
            Poiché il nome utente è obbligatorio e unico sulla piattaforma (e non fornito da Google), il sistema deve reindirizzare l'utente a un modulo di completamento dove dovrà inserire solo il nome utente desiderato.
            L'account viene creato e attivato immediatamente (l'e-mail è già verificata da Google).

            \rfItem \textbf{Visualizzazione Dashboard:} 
            La pagina principale deve mostrare un feed di attività rilevanti.
            Il feed deve mostrare un elenco di partite e sfide disponibili, ordinate cronologicamente (dalla data più vicina).
            L'utente deve poter filtrare questo feed in base a sport, luogo e contesto ("Partite create da me"). La pagina deve contenere un widget per visualizzare il proprio profilo e le statistiche principali (numero di amici, numero di partite giocate, posizione in classifica)

            \rfItem \textbf{Ricerca globale:} 
            Il sistema deve fornire una funzionalità di ricerca (tramite una barra di ricerca nell'header).
            L'utente inserisce una stringa di testo, il sistema deve cercare la stringa nei seguenti ambiti: utenti (per nome utente), luoghi (per nome struttura) e partite (per titolo o sport).
            Il sistema deve presentare i risultati in un pannello (popup), raggruppati per categoria (utenti trovati, luoghi trovati, ecc.).
            L'utente deve poter utilizzare un menu a tendina per restringere la ricerca a una sola categoria (es. "Cerca solo utenti").

            \rfItem \textbf{Creazione Partita:} 
            Un utente deve poter creare una nuova partita, per farlo compila un form che richiede: sport (selezionato da un elenco predefinito), luogo (selezionato tramite la mappa), data e orario, tipo di partita (individuale / a squadre), visibilità (pubblica, privata), note aggiuntive (campo di testo libero).

            \rfItem \textbf{Visualizzazione mappa delle strutture:} 
            Il sistema deve fornire una mappa interattiva con degli indicatori in corrispondenza delle strutture sportive pubbliche, cliccando su una struttura si apre una scheda informativa (popup) sulla struttura. L'utente deve poter filtrare le strutture visualizzate in base allo sport.

            \rfItem \textbf{Visualizzazione classifica:}
            Il sistema deve mostrare una classifica degli utenti basata sui risultati delle partite.
            La classifica deve essere filtrabile in base allo sport e in base al gruppo di utenti (classifica globale / solo amici).

            \rfItem \textbf{Visualizzazione profilo utente:}
            Ogni utente deve avere una pagina profilo pubblica che mostra: l'elenco degli amici, un grafico dell'andamento delle ultime partite, un calendario delle partite passate e future.
            \newline
            Quando un utente visualizza il profilo di un altro utente deve avere due bottoni: "Invia richiesta di amicizia" e "Segnala utente"

            \rfItem \textbf{Partecipazione a partite pubbliche:}
            Il sistema deve permettere a un utente di unirsi a partite create da altri.
            Quando un utente visualizza una partita disponibile (dal feed, dalla mappa o dalla ricerca), deve poterne vedere i dettagli e unirsi tramite il pulsante "Unisciti".

            \rfItem \textbf{Gestione richieste di amicizia:}
            Il sistema deve permettere a un utente di gestire le richieste di amicizia in entrata.
            L'utente riceve una notifica che lo reindirizza alla sezione "Amici" (raggiungibile anche dalla sezione "Profilo utente") dove visualizza un elenco di richieste di amicizia, l'utente può accettare o rifiutare le richieste.

            \rfItem \textbf{Segnalazione utente:}
            Il sistema deve fornire un meccanismo per garantire la sicurezza e il corretto comportamento degli utenti. Sul profilo di ogni utente deve essere presente un pulsante "Segnala", cliccandolo si apre un popup che chiede: motivazione (selezionata tra un elenco predefinito) e descrizione (per fornire dettagli aggiuntivi)
        \end{rfList}
    \section{Requisiti funzionali per i creatori di sfide}
        \begin{rfList}
            \rfItem \textbf{Inserimento risultati:}
            Al termine dell'orario previsto per una partita, l'utente deve poter inserire i risultati per aggiornare le statistiche. L'inserimento dei risultati deve essere abilitato solo dopo la conclusione della partita. Il risultato sarà inserito dal creatore della partita.
        \end{rfList}
    \section{Requisiti funzionali per gli addetti del comune}
        \begin{rfList}
            \rfItem \textbf{Visualizzazione statistiche:} 
            Il sistema deve permettere agli utenti comunali (che avranno un account "speciale") di visualizzare una pagina con: lista degli sport più praticati e relativo grafico nel tempo, strutture più utilizzate e relativo grafico nel tempo.
        \end{rfList}     
    \section{Requisiti funzionali per gli amministratori}
        \begin{rfList}
            \rfItem \textbf{Revisione delle strutture} 
            Il sistema deve permettere agli amministratori di aggiungere, eliminare o modificare informazioni riguardanti le strutture sportive pubbliche.
        \end{rfList}
    