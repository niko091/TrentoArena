\chapter{Use Case Diagram}
\section{RF1: Fare login \newline
RF2: Registrare nuovo utente}

\begin{figure}[h]
    \centering
    \includegraphics[width=0.5\textwidth]{UseCase/LoingRegistrazione.png}
\end{figure}

\textbf{Use Case RF1: fare login}
\newline Come l'utente anonimo fa il login nel sito. \newline
\textbf{Descrizione: }
\begin{enumerate}
    \item L'utente visualizza una schermata con un form.
    \item L'utente inserisce all'interno degli appositi campi email o il nome utente e la password, preme sul pulsante “Login” presente nella schermata e se email/nome utente e password sono corretti, si apre la schermata principale dell'applicazione [eccezione 1].
    \item Se l'utente seleziona il login di Google, l'utente dovrà inserire le proprie credenziali di Google in altre apposite schermate e quindi, se nome utente e password sono corretti, si apre la schermata principale dell'applicazione [eccezione 1].
\end{enumerate}
\textbf{Eccezioni:}
\begin{enumerate}
    \item Se l'utente inserisce una email/nome utente o password sbagliata, l'applicazione restituisce un messaggio di errore e chiede nuovamente di effettuare il login.
\end{enumerate}
\textbf{Use Case RF2: registrare nuovo utente} \newline
Come l'utente anonimo si registra nel sito.
\begin{enumerate}
    \item L'utente visualizza una schermata nella quale inserire alcuni dati obbligatori: nome, cognome, data di nascita, indirizzo mail e nome utente [eccezione 1]
    \item L'utente deve inserire una password e confermare che l'inserimento sia corretto reimmettendola [eccezione 2].
    \item Per confermare l'indirizzo di posta elettronica, l'utente riceve una email dalla piattaforma, nella quale è presente un link di verifica [eccezione 3].
    \item Se l'utente seleziona la registrazione di Google, l'utente dovrà inserire le proprie credenziali di Google.
    \item Per inserire il nome utente dopo la registrazione di Google, l'utente viene reindirizzato ad un modulo di completamento dove dovrà inserire solo il nome utente desiderato.
    \item Per accedere alla piattaforma, si che l'utente abbia fatto la registrazione con Google o quella standard dovrà fare il login [eccezione 1].
\end{enumerate}
\textbf{Eccezioni:}
\begin{enumerate}
    \item Il nome utente deve essere unico al interno delle piattaforma, se l'utente inserisce un nome già in uso, l'applicazione ritorna un messaggio di errore e l'utente deve reinserire un nickname diverso.
    \item Se la password inserita nel campo conferma password è diversa da quella inserita nel campo password, la applicazione ritorna un messaggio di errore e l'utente deve reinserire la password.
    \item Se l'utente non verifica l'email il suo account non può essere creato, l'utente può farsi rinviare l'email o cambiare l'indirizzo di posto elettronica.
\end{enumerate}