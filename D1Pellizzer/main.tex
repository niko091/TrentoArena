\documentclass[twoside, a4paper, 10pt]{report}
\usepackage[italian]{babel}
\usepackage[utf8]{inputenc}
\usepackage[margin=1in]{geometry}
\usepackage{graphicx}
\usepackage{fancyhdr}
\usepackage{array}
\usepackage{colortbl}
\usepackage{lastpage}
\usepackage{titlesec}
\usepackage{float}
\usepackage{subcaption}
\usepackage{hyperref}
\usepackage{afterpage}

% Ridefinizione per il titolo dei capitoli
\titleformat{\chapter}[hang]{\LARGE\bfseries}{\thechapter}{1em}{} 
\titlespacing{\chapter}{0pt}{0pt}{1em}

% Definizione della path per le immagini
\graphicspath{{../images/}}

% Set the version of the document
\newcommand{\version}{1.0} 
\newcommand{\ProjectTitle}{TrentoArena}
\newcommand{\ProjectTitleShort}{TrentoArena}
\newcommand{\FileName}{D1-\ProjectTitleShort-descrizioneProgetto}

% Definizione dei dati del documento
\title{Descrizione di Progetto - \ProjectTitle}
\author{Bancher Nicola, Bignotti Giorgio, Pellizzer Mattia}
\date{A.A. 2025/2026}

% Definizione metadati PDF
\hypersetup{
    pdftitle={\ProjectTitle},
    pdfauthor={Bancher Nicola, Bignotti Giorgio, Pellizzer Mattia},
    pdfsubject={Descrizione di Progetto},
    pdfkeywords={\ProjectTitle, Descrizione di Progetto, Documento di Analisi, Comune di Trento, UniTN}
}

% Definizione counter Requisiti Funzionali
\newcounter{rfCounter}
\newcounter{rnfCounter}

% Definisci un nuovo comando per il formato RF/RNF
\newcommand{\RF}{RF\arabic{rfCounter}}
\newcommand{\RNF}{RNF\arabic{rnfCounter}}

% Definizione nuovo comando per lista con RF/RNF automatici in modo che NON si resettino ad ogni lista
% Ambiente per la lista di RF
\newenvironment{rfList}{
    \begin{list}{\textbf{\RF:}}{ \setlength{\itemsep}{0pt} } % Lista RF
        \setcounter{rfCounter}{\value{rfCounter}} % Mantieni il valore corrente
}{\end{list}}

% Ambiente per la lista di RNF
\newenvironment{rnfList}{
    \begin{list}{\textbf{\RNF:}}{ \setlength{\itemsep}{0pt} } % Lista RNF
        \setcounter{rnfCounter}{\value{rnfCounter}} % Mantieni il valore corrente
}{\end{list}}

% comandi per gli item delle liste RF e RNF
\newcommand{\rfItem}{\stepcounter{rfCounter}\item}
\newcommand{\rnfItem}{\stepcounter{rnfCounter}\item}

% Rimozione scritta "Capitolo" dai titoli dei capitoli
\renewcommand{\chaptermark}[1]{%
    \markboth{
        \thechapter.\ #1%
    }{}%
}
% Definizione del layout della pagina
\fancypagestyle{stdPage}{
    \setlength{\headheight}{24.0pt} 
    \renewcommand{\footrulewidth}{0.4pt}
    \fancyhead{}
    \fancyfoot{}
    \fancyhead[LE,RO]{\begin{tabular}{l l}
        \textbf{Document:} & Descrizione di progetto \\
        \textbf{Version:} & \version
    \end{tabular}}
    \fancyfoot[LE,RO]{\thepage / \pageref*{LastPage}}
    \fancyhead[LO,RE]{\leftmark}
}
\fancypagestyle{plain}{
    \pagestyle{stdPage}
}
\fancypagestyle{index}{
    \pagestyle{stdPage}
    \fancyfoot[LE,RO]{\thepage}
}

\fancypagestyle{emptyPage}{
    \setlength{\headheight}{24.0pt} 
    \renewcommand{\headrulewidth}{0pt}
    \fancyhead{}
    \fancyfoot{}
}

% Definizione della pagina bianca
\newcommand\blankpage{%
    \null
    \thispagestyle{empty}%
    \newpage}

\begin{document}
    \pagestyle{fancy}
    \pagenumbering{Roman} 
    
    \begin{titlepage}
        \thispagestyle{emptyPage}
        \textbf{Progetto:}
        \vspace{0.5cm}
        \begin{center}
            \textbf{\Huge{\ProjectTitle}}
        \end{center}
        \vspace{1cm}
        \textbf{Titolo del documento:}
        \vspace{0.5cm}
        \begin{center}
            \textbf{\huge{Descrizione di Progetto}}
        \end{center}
        \vspace{1cm}
        \textbf{Document Info}
        \vspace{0.5cm}
        % Table with document info
        \begin{center}
            \begin{tabular}{|l|l|l|c|}  
                \hline
                {\cellcolor[rgb]{0,0.502,1}}\textcolor{white}{\textbf{Doc. Name}}   & \FileName & {\cellcolor[rgb]{0,0.502,1}}\begin{tabular}[c]{@{}>{\cellcolor[rgb]{0,0.502,1}}l@{}}\textcolor{white}{\textbf{Doc.}}\\\textcolor{white}{\textbf{Number}}\end{tabular} & D1 V\version  \\ 
                \hline
                {\cellcolor[rgb]{0,0.502,1}}\textcolor{white}{\textbf{Description}} & \multicolumn{3}{l|}{Documento di analisi dei requisiti funzionali, non funzionali e front-end}                                                                                                                               \\
                \hline
            \end{tabular}
        \end{center}
        % Document authors (1 per line) with name and ID aligned to the right but with some space from the right border 
        \vspace{1.5in}
        \vfill
        \begin{flushright}
            \rightskip=2cm
            \begin{tabular}{r l}
                \multicolumn{2}{c}{\textbf{Authors}} \\
                Bancher Nicola   & 252781 \\
                Bignotti Giorgio & 254516 \\
                Pellizzer Mattia & 253833
            \end{tabular}
        \end{flushright}
        \vfill
    \end{titlepage}
    % \afterpage{\blankpage} % Uncomment this line if to print as a booklet
    \begingroup
        \setcounter{tocdepth}{0}
        \tableofcontents
        \thispagestyle{index}
    \endgroup
    \pagestyle{stdPage}
    % \afterpage{\blankpage} % Uncomment this line if to print as a booklet
    \newpage
    \pagenumbering{arabic} 
    
    \chapter{Il progetto \ProjectTitle}
% Grid with 1 row and 3 columns for images inside the folder slides
\begin{figure}[H]
    \begin{subfigure}{0.33\textwidth}
        \includegraphics[width=\textwidth]{slides/Diapositiva2.PNG}
        \caption{I problemi}
    \end{subfigure}
    \begin{subfigure}{0.33\textwidth}
        \includegraphics[width=\textwidth]{slides/Diapositiva3.PNG}
        \caption{La soluzione}
    \end{subfigure}
    \begin{subfigure}{0.33\textwidth}
        \includegraphics[width=\textwidth]{slides/Diapositiva4.PNG}
        \caption{I vantaggi}
    \end{subfigure}
\end{figure}

\section{I problemi}
    Il progetto mira a risolvere tre problemi:\newline
    Il primo problema riguarda uno scarso senso di comunità sportiva all'interno Comune di Trento, che essendo una città universitaria ospita molte persone con background differenti, molte delle quali praticano sport a livello amatoriale, ma si limitano e incotrarsi in piccoli gruppi o non riescono a trovare altri con la stessa passione.
    \newline
    Il secondo problema riguarda la difficoltà  nell'organizzare incontri sportivi amatoriali, gli strumenti per organizzarsi sono molti, come gruppi su app di messaggistica, pagine sui social, volantini e passaparola. Le strutture sportive pubbliche sono molte, ben attrezzate e distribuite in tutto il territorio. Questo comporta una frammentazione della comunità sportiva.
    \newline
    Il terzo problema riguarda la mancanza di motivazione e costanza per continuare ad allenarsi, quando si inizia a praticare uno sport a livello amatoriale si è motivati dal fatto che si sta iniziando una nuovo percorso, ma col passare del tempo questa tende a svanire, soprattutto senza stimoli esterni e qualcuno coi cui condividere la stessa passione.
\section{La soluzione}
   Il progetto mira a realizzare un'applicazione web che permetta ai cittadini di organizzare incontri sportivi amatoriali per gli sport più popolari nei quali ci sono due giocatori/squadre, nei quali c'è una chiara definizione di vittoria/sconfitta e per cui il comune offre delle strutture pubbliche. Dopo aver effettuato l'accesso, sarà possibile lanciare/partecipare a sfide, visualizzare gli impianti tramite la mappa, inviare richieste di amicizia per vedere le attività degli altri utenti e competere per scalare la classifica globale. Tutte le funzionalità dell'applicazione saranno completamente fruibili tramite browser, senza necessità di installare l'app in locale. L'applicazione offrirà funzionalità come notifiche per eventi creati dai tuoi amici o di tuo interesse.
   
\section{I vantaggi}
    \begin{itemize}
        \item \textbf{Semplcità nel trovare/creare partite:} la possibilità di trovare o creare sfide in modo semplice incoraggia gli utenti a mettersi in gioco e partecipare a partite già esistenti.
        \item \textbf{Gamification:} tramite il meccanismo di punti e livelli (che funziona in modo simile all'ELO negli scacchi) gli utenti sono invogliati a partacipare e a sfidarsi tra di loro per scalare la classifica e aumentare il proprio status virtuale.
        \item \textbf{Visualizzazione deli impianti pubblici:} grazie alla mappa interattiva è possibile visualizzare la posizione e lo stato delle strutture sportive pubbliche, insieme alla lista delle attrezzature presenti e alle sfide che si terranno in quella struttura.
        \item \textbf{Interfaccia intuitiva:} Presentare le informazioni con un'interfaccia grafica, accessibile anche a utenti sprovvisti di conoscenze informatiche, assicura che l'applicazione sarà utile a una vasta quantità di cittadini.
        \item \textbf{raccolta di dati per il comune:} i dati relativi alla frequenza di utilizzo delle strutture pubbliche, all'affluenza nei diversi giorni/orari e alla pololarità dei diversi sport, saranno resi disponibili al comune (in forma anonimizzata).
    \end{itemize}


\section{I limiti dell'applicazione}
    \begin{itemize}
        \item \textbf{Dipendeza da una connesione internet:} per visualizzare le sfide create da altri utenti, lo stato delle strutture e la classifica è necessario essere connessi a internet.
        \item \textbf{Il sistema si basa sull’onestà degli utenti:} la vittoria deve essere registrata nell'app dagli utenti, che quindi dovranno stabilire un vincitore. Questa decisione non può essere presa direttamente dall'applicazione, ma si basa sull'onestà degli utenti.
        \item \textbf{L’incontro tra utenti non conosciuti può esporre a rischi o disagi:} incontrare dal vivo persone conosciute tramite app può portare a disagi/rischi, l'applicazione cerca di raccogliere feedback sul comportamento dei giocatori, coloro che verranno segnalati come non adatti verranno sospesi temporaneamente e dopo un certo numero di sospensioni verranno banditi.
        \item \textbf{L’app richiede un’utenza attiva per mantenere la sua utilità:} l'app richiede una discreta partecipazione da parte degli utenti, in quanto gli eventi vengono creati dalla community.
    \end{itemize}
    \chapter{Requisiti Funzionali} 
    \section{Requisiti funzionali comuni a tutti gli utenti}
        \begin{rfList}
            \rfItem \textbf{Login:} 
            L'utente deve potersi autenticare per accedere alla piattaforma. Il sistema deve fornire due metodi di accesso:
            \newline
            \textbf {Login Standard:} 
            L'utente inserisce la propria e-mail (o nome utente) e password. Il sistema verifica le credenziali; in caso di successo, reindirizza l'utente alla propria Dashboard
            \newline
            \textbf{Login tramite Google:} 
            L'utente seleziona "Accedi con Google". Il sistema avvia il flusso di autenticazione OAuth2. Se l'utente Google è già associato a un account della piattaforma, il sistema lo autentica e lo reindirizza alla propria dashboard, altrimenti vedi RF2.

            \rfItem \textbf{Registrazione:} 
            Il sistema deve permettere a un nuovo utente di creare un account. Sono previsti due flussi:
            \newline
            \textbf {Registrazione Standard:} 
            L'utente compila un modulo che richiede: nome utente (che deve essere unico nel sistema), e-mail, password e conferma password.
            Il sistema valida i dati (unicità del nome utente, validità dell'email, corrispondenza delle password). Il sistema crea l'account in stato "non confermato" e invia un'e-mail di conferma all'indirizzo fornito. L'utente deve cliccare sul link nell'e-mail per attivare l'account
            \newline
            \textbf{Registrazione tramite Google:} 
            L'utente seleziona "Registrati con Google".
            Il sistema recupera l'e-mail e il nome dall'account Google.
            Poiché il nome utente è obbligatorio e unico sulla piattaforma (e non fornito da Google), il sistema deve reindirizzare l'utente a un modulo di completamento dove dovrà inserire solo il nome utente desiderato.
            L'account viene creato e attivato immediatamente (l'e-mail è già verificata da Google).

            \rfItem \textbf{Visualizzazione Dashboard:} 
            La pagina principale deve mostrare un feed di attività rilevanti.
            Il feed deve mostrare un elenco di partite e sfide disponibili, ordinate cronologicamente (dalla data più vicina).
            L'utente deve poter filtrare questo feed in base a sport, luogo e contesto ("Partite create da me"). La pagina deve contenere un widget per visualizzare il proprio profilo e le statistiche principali (numero di amici, numero di partite giocate, posizione in classifica)

            \rfItem \textbf{Ricerca globale:} 
            Il sistema deve fornire una funzionalità di ricerca (tramite una barra di ricerca nell'header).
            L'utente inserisce una stringa di testo, il sistema deve cercare la stringa nei seguenti ambiti: utenti (per nome utente), luoghi (per nome struttura) e partite (per titolo o sport).
            Il sistema deve presentare i risultati in un pannello (popup), raggruppati per categoria (utenti trovati, luoghi trovati, ecc.).
            L'utente deve poter utilizzare un menu a tendina per restringere la ricerca a una sola categoria (es. "Cerca solo utenti").

            \rfItem \textbf{Creazione Partita:} 
            Un utente deve poter creare una nuova partita, per farlo compila un form che richiede: sport (selezionato da un elenco predefinito), luogo (selezionato tramite la mappa), data e orario, tipo di partita (individuale / a squadre), visibilità (pubblica, privata), note aggiuntive (campo di testo libero).

            \rfItem \textbf{Visualizzazione mappa delle strutture:} 
            Il sistema deve fornire una mappa interattiva con degli indicatori in corrispondenza delle strutture sportive pubbliche, cliccando su una struttura si apre una scheda informativa (popup) sulla struttura. L'utente deve poter filtrare le strutture visualizzate in base allo sport.

            \rfItem \textbf{Visualizzazione classifica:}
            Il sistema deve mostrare una classifica degli utenti basata sui risultati delle partite.
            La classifica deve essere filtrabile in base allo sport e in base al gruppo di utenti (classifica globale / solo amici).

            \rfItem \textbf{Visualizzazione profilo utente:}
            Ogni utente deve avere una pagina profilo pubblica che mostra: l'elenco degli amici, un grafico dell'andamento delle ultime partite, un calendario delle partite passate e future.
            \newline
            Quando un utente visualizza il profilo di un altro utente deve avere due bottoni: "Invia richiesta di amicizia" e "Segnala utente"

            \rfItem \textbf{Partecipazione a partite pubbliche:}
            Il sistema deve permettere a un utente di unirsi a partite create da altri.
            Quando un utente visualizza una partita disponibile (dal feed, dalla mappa o dalla ricerca), deve poterne vedere i dettagli e unirsi tramite il pulsante "Unisciti".

            \rfItem \textbf{Gestione richieste di amicizia:}
            Il sistema deve permettere a un utente di gestire le richieste di amicizia in entrata.
            L'utente riceve una notifica che lo reindirizza alla sezione "Amici" (raggiungibile anche dalla sezione "Profilo utente") dove visualizza un elenco di richieste di amicizia, l'utente può accettare o rifiutare le richieste.

            \rfItem \textbf{Segnalazione utente:}
            Il sistema deve fornire un meccanismo per garantire la sicurezza e il corretto comportamento degli utenti. Sul profilo di ogni utente deve essere presente un pulsante "Segnala", cliccandolo si apre un popup che chiede: motivazione (selezionata tra un elenco predefinito) e descrizione (per fornire dettagli aggiuntivi)
        \end{rfList}
    \section{Requisiti funzionali per i creatori di sfide}
        \begin{rfList}
            \rfItem \textbf{Inserimento risultati:}
            Al termine dell'orario previsto per una partita, l'utente deve poter inserire i risultati per aggiornare le statistiche. L'inserimento dei risultati deve essere abilitato solo dopo la conclusione della partita. Il risultato sarà inserito dal creatore della partita.
        \end{rfList}
    \section{Requisiti funzionali per gli addetti del comune}
        \begin{rfList}
            \rfItem \textbf{Visualizzazione statistiche:} 
            Il sistema deve permettere agli utenti comunali (che avranno un account "speciale") di visualizzare una pagina con: lista degli sport più praticati e relativo grafico nel tempo, strutture più utilizzate e relativo grafico nel tempo.
        \end{rfList}     
    \section{Requisiti funzionali per gli amministratori}
        \begin{rfList}
            \rfItem \textbf{Revisione delle strutture} 
            Il sistema deve permettere agli amministratori di aggiungere, eliminare o modificare informazioni riguardanti le strutture sportive pubbliche.
        \end{rfList}
    
    \chapter{Requisiti Non Funzionali} 
    \begin{rnfList}
        \rnfItem \textbf{Latenza} 
        Il sistema deve garantire un'esperienza utente fluida. I tempi di caricamento non devono superare i 2 secondi in condizioni di carico normale. Le azioni utente (es. creazione partita, invio richiesta amicizia) devono ricevere un feedback visivo (conferma o errore) entro 1 secondo.
        \rnfItem \textbf{Multi Utenza}
        L'architettura del sistema deve essere in grado di supportare almeno 200 utenti connessi simultaneamente che eseguono operazioni senza degradazione delle performance definite nel RNF1.
        \rnfItem \textbf{Sicurezza Dati}
        Tutta la comunicazione tra il client (browser) e il server deve essere crittografata tramite protocollo HTTPS. Le password degli utenti devono essere salvate nel database esclusivamente in forma di hash. I dati personali sensibili devono essere protetti adeguatamente nel database.
        \rnfItem \textbf{Prevenzione Vulnerabilità}
        L'applicazione deve essere protetta dalle vulnerabilità web più comuni. Deve essere implementata una validazione robusta di tutti gli input utente, sia lato client che lato server.
        \rnfItem \textbf{GDPR}
        Il sistema deve essere conforme al Regolamento Generale sulla Protezione dei Dati (GDPR).
        \rnfItem \textbf{Backup}
        Deve essere in atto un piano di backup locale (su un server SFTP locale con frequenza giornaliera) e remoto (con frequenza settimanale per protezione da disastri)
        \rnfItem \textbf{Facilità d'uso} 
        L'interfaccia utente deve essere intuitiva e coerente. Un nuovo utente deve essere in grado di completare i flussi principali senza necessità di tutorial. Il sistema deve fornire feedback chiari per ogni azione.
        \rnfItem \textbf{Compatibilità browser}
        L'applicazione deve essere pienamente funzionale e visivamente coerente sulle versioni più recenti dei principali browser: Chrome 80+, Firefox 75+, Safari 13+ ed Edge 80.
        \rnfItem \textbf{Multilingua} 
        Il sistema deve supportare la multilingua: italiano (default), inglese e tedesco.
    \end{rnfList}

    \chapter{Use Case Diagram}
\section{RF1: Fare login \newline
RF2: Registrare nuovo utente}

\begin{figure}[h]
    \centering
    \includegraphics[width=0.5\textwidth]{UseCase/LoingRegistrazione.png}
\end{figure}

\textbf{Use Case RF1: fare login}
\newline Come l'utente anonimo fa il login nel sito. \newline
\textbf{Descrizione: }
\begin{enumerate}
    \item L'utente visualizza una schermata con un form.
    \item L'utente inserisce all'interno degli appositi campi email o il nome utente e la password, preme sul pulsante “Login” presente nella schermata e se email/nome utente e password sono corretti, si apre la schermata principale dell'applicazione [eccezione 1].
    \item Se l'utente seleziona il login di Google, l'utente dovrà inserire le proprie credenziali di Google in altre apposite schermate e quindi, se nome utente e password sono corretti, si apre la schermata principale dell'applicazione [eccezione 1].
\end{enumerate}
\textbf{Eccezioni:}
\begin{enumerate}
    \item Se l'utente inserisce una email/nome utente o password sbagliata, l'applicazione restituisce un messaggio di errore e chiede nuovamente di effettuare il login.
\end{enumerate}
\textbf{Use Case RF2: registrare nuovo utente} \newline
Come l'utente anonimo si registra nel sito.
\begin{enumerate}
    \item L'utente visualizza una schermata nella quale inserire alcuni dati obbligatori: nome, cognome, data di nascita, indirizzo mail e nome utente [eccezione 1]
    \item L'utente deve inserire una password e confermare che l'inserimento sia corretto reimmettendola [eccezione 2].
    \item Per confermare l'indirizzo di posta elettronica, l'utente riceve una email dalla piattaforma, nella quale è presente un link di verifica [eccezione 3].
    \item Se l'utente seleziona la registrazione di Google, l'utente dovrà inserire le proprie credenziali di Google.
    \item Per inserire il nome utente dopo la registrazione di Google, l'utente viene reindirizzato ad un modulo di completamento dove dovrà inserire solo il nome utente desiderato.
    \item Per accedere alla piattaforma, si che l'utente abbia fatto la registrazione con Google o quella standard dovrà fare il login [eccezione 1].
\end{enumerate}
\textbf{Eccezioni:}
\begin{enumerate}
    \item Il nome utente deve essere unico al interno delle piattaforma, se l'utente inserisce un nome già in uso, l'applicazione ritorna un messaggio di errore e l'utente deve reinserire un nickname diverso.
    \item Se la password inserita nel campo conferma password è diversa da quella inserita nel campo password, la applicazione ritorna un messaggio di errore e l'utente deve reinserire la password.
    \item Se l'utente non verifica l'email il suo account non può essere creato, l'utente può farsi rinviare l'email o cambiare l'indirizzo di posto elettronica.
\end{enumerate}
    \include{chapters/05-userStory}
    \include{chapters/06-DesignFrontEnd}

\end{document}